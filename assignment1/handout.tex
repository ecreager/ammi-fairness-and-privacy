\documentclass{article}

\usepackage{amssymb}

\begin{document}

\begin{center}

{\bf AMMI Privacy and Fairness Course, Rwanda, May 2019}\\

{\bf Assignment 1}\\

\end{center}

Answer three out of the following four questions.

Please turn in by Noon on Thursday, May 14.

\begin{itemize}

\item[1a.] Prove that 
the following two definitions of $(\epsilon,0)$-DP are the same.

\begin{itemize}

\item For every two neighboring databases $x,y$ and
for each element $r \in R$,
$Pr[M(x)=r] \leq e^{\epsilon} Pr[M(y)=r] $.


\item For every two neighboring databases $x,y$, and for every
subset $S \subseteq R$, $Pr[M(x) \in S] \leq e^{\epsilon} Pr[M(y)\in S] .$

\end{itemize}

\item[1b.] What happens in the case of $(\epsilon,\delta)$-DP?

\item[2.] Prove that if $M_1,...,M_k$ are $(\epsilon,\delta)$-DP mechanisms,
then any convex combination is also a $(\epsilon,\delta)$-DP mechanism.
        (A convex combination mechanism $M$ on database $x$ first picks $i \in \{1\ldots k\}$
according to some distribution over $\{1\ldots k\}$, and then runs
mechanism $M_i$ on $x$. Thus the distribution of output values of
$M(x)$ is a convex combination of the distributions of output values
        of $M_1(x),....,M_k(x)$.)

%(A convex combination is defined by a distribution $p$ over $\{1,\ldots,k\}$. 
%The
%convex combination mechanism first picks $i \in [k]$ according to $p$,
%and then runs mechanism $M_i$ on $x$.)

\item[3.] Prove that any mechanism $M$ that is deterministic
is not differentially private.

\item[4a.] (Group Privacy.)
Let $M$ be a mechanism mapping $\mathbb{N}^{|X|}$ to $R$,
Prove that any $(\epsilon,0)$-DP mechanism M is
$(k\epsilon,0)$-DP
for groups of size $k$ i.e. for all $x,y$ such that
$||x - y||_1 \leq k$,
$Pr[M(x) \in S] \leq e^{\epsilon k} Pr[M(y) \in S]$.

\item[4b] Prove the following approximate group
privacy property. Any $(\epsilon,\delta)$-DP mechanism $M$
is $(k\epsilon, k \cdot e^{k\cdot \epsilon} \cdot \delta)$-DP
for groups of size $k$. Note that both $\epsilon$ and $\delta$ are nonnegative.

\end{itemize}

\end{document}
